\section{Performances}

\subsection{Taux de réussite}
Un premier calcul intéressant à réaliser est celui du taux de réussite. Nous connaissons les classes réelles des caractères et celles déterminées par le 
programme : il suffit de les comparer afin d'obtenir un pourcentage de réussite.
\\
\begin{lstlisting}
reussite = 0;
for i=1:10
	for j=1:10
		if  res(i,j) == i-1
			reussite = reussite + 1;
		end
	end
end
\end{lstlisting}

Dans notre cas, nous obtenons une réussite de 84 \%.

\subsection{Matrice de confusion}

\paragraph{}
Une matrice de confusion nous permet de savoir les erreurs commises par notre programme Matlab. En effet, grâce à elle, nous pouvons visualiser quelles sont 
les classes confondues par d'autres. Dans cette matrice, chaque valeur C(i, j) correspond au nombre de caractère de la classe i ont été confondu avec un. 
caractère de la classe j. Pour un taux de réussite optimal, seule la diagonale de cette matrice doit posséder des coefficients non-nuls.
\\
\begin{lstlisting}
confusion = zeros(10,10);
for i=1:10
    for j=1:10
        confusion(i,j) = length(find(res(i,:) == j-1))*10;
    end
end
\end{lstlisting}

\begin{lstlisting}
confusion = 
	80     0     0     0     0     0     0     0     0    20
	 0    90     0     0     0     0     0     0     0    10
	 0     0   100     0     0     0     0     0     0     0
	 0     0     0    90     0     0     0     0    10     0
	 0     0     0     0    80    10    10     0     0     0
	 0     0     0     0     0   100     0     0     0     0
	 0     0     0     0     0     0    80    20     0     0
	 0     0     0    10     0     0     0    90     0     0
	10     0    10    10     0     0     0     0    70     0
	 0     0    20     0     0     0     0     0    20    60
\end{lstlisting}
