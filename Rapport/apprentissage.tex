\section{Apprentissage}

\paragraph{}
Cette étape consisite à fournir au programme un jeu de caractères où chaque caractère est identifié : le programme sait s'il a affaire à un 0, un 1, etc. 
Ce jeu va lui permettre de reconnaître les caractéristiques de chaque caractère et donc de les reconnaître par la suite. La partie du programme extrayant 
les différents caractères de l'image est fournie par le sujet. 

\paragraph{}
Nous avons choisi de nous intéresser à la méthode du zoning utilisant les densités de pixels. Pour cela, nous avons développé la fonction extraitDensite 
ci-dessous.
Nous avons traité le problème de la taille des imagettes en ajoutant du blanc à droite et en bas de l'image afin d'obtenir des dimensions nous permettant de définir les zones souhaitées.
Notre programme a rapporté de meilleurs résultats en découpant nos images avec une grille carrée de dimension 6, en testant avec des valeurs allant de 4 à 9.
\\
\begin{lstlisting}
function [ densites ] = extraitDensite(img, n, m)
	% On cree un vecteur qui contiendra les densites de chaque zone.
	densites = zeros(n*m, 1);
	% On cree une seconde image dont les dimensions sont corrigees afin de correspondre au nombre de zones demandees.
	[lignes1, colonnes1] = size(img);
	pasLignes = ceil(lignes1/n);
	pasColonnes = ceil(colonnes1/m);
	% On remplit cette image de blanc.
	imgCor = ones(pasLignes*n, pasColonnes*m);
	imgCor = 255*imgCor;
	% On copie les valeurs de l'image initiale dans la nouvelle.
	for i=1:lignes1
		for j=1:colonnes1
			imgCor(i,j) = img(i, j);
		end
	end
	% Initialisation des compteurs pour le calcul de la densite.
	i = 1;
	j = 1;
	iDen = 1;
	for i=1:pasLignes:pasLignes*n
		for j=1:pasColonnes:pasColonnes*m
			% On extrait la zone de l'image a traiter.
			img2 = subimage(imgCor, pasColonnes-1, pasLignes-1, j, i);
			% Calcul de la densite.
			densites(iDen) = sum(sum(img2)) / (pasColonnes*pasLignes*255);
			iDen = iDen+1;
		end
	end
end
\end{lstlisting}

\paragraph{}
Cette fonction retourne un vecteur de taille n * m contenant les densités de chacune des n * m zones de l'image. Pour les calculer, une deuxième image 
est créée dont les dimensions permettent une division en n * m zones. Pour cela, des pixels blancs sont ajoutés en bas et à droite de l'image. Cette pratique 
cause une légère perte de précision du calcul, mais le simplifie notablement.

\paragraph{}
On applique ensuite la fonction subimage pour découper la zone de l'image à traiter, et on calcule la densité de cette zone par une simple somme des pixels 
de la zone. Cette densité est ensuite normalisée par une division par la surface de la zone multipliée par 255.

\paragraph{}
Ce vecteur est ensuite stocké dans la matrice densites au rang correspondant au caractère traité. Cette matrice est enfin stockée dans le disque dur pour 
être réutilisée par la partie décisionnaire du programme.






