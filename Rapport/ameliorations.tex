\section{Améliorations possibles}

\paragraph{}
Après analyse de la matrice de confusion, nous avons envisagé l'implémentation d'un perceptron mono-couche permettant d'indiquer si une imagette correspond effectivement à une chiffre ou non. Nous pouvons ainsi confirmer 
la reconnaissance d'un caractère, ou bien utiliser le second voisin le plus proche en cas de non validation par le perceptron. Cette méthode permettrait ainsi d'éliminer certains cas erronés issus de la méthode du KPPV.

\paragraph{}
Cependant, le KPPV conserve des propriétés contraignantes telles que des temps de traitement extrêmement longs lorsqu'il est question de manier de grands échantillons. Une autre piste à explorer pour améliorer la phase de reconnaissance pourrait donc être l'adoption de la méthode de la distance euclidienne. Celle-ci, s'appuyant sur des moyennes, semble être moins lourde que la méthode du KPPV.
\\




