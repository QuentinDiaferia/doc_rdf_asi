\section{Prise de décision}

Pour l'étape de décision de notre programme, nous nous sommes penchés vers la méthode simple des K plus proches voisins. Nous avons donc développé la 
fonction kppv suivante :
\\
\begin{lstlisting}
function [ classe ] = kppv( densitePoint, matDensites, k )
    distance = zeros(200,1);
    for i=1:200
        tmp = 0;
        for j=1:25
            tmp = tmp + abs(densitePoint(j) - matDensites(i,j));
        end
        distance(i) = tmp;
    end
    plusProches = zeros(k, 1);
    for i=1:k
       indice = find(distance == min(distance));
       plusProches(i) = floor(indice / 20);
       distance(indice, :) = max(distance);
    end
    sort(plusProches);
    classe = 0;
    for i=0:9
        if(nnz(plusProches == i) >  classe)
             classe = i;
        end;
    end
end
\end{lstlisting}


Nous avons enregistré de meilleurs résultats en étant assez stricts sur le voisinage. (Le paramètre k étant alors faible, égal à 1)