\section{Conclusion}

\paragraph{}
Malgré un taux de reconnaissance passable et des délais d'exécution corrects, la méthode du KPPV retenue peut se voir améliorer afin d'optimiser les performances du programme. Aussi, l'utilisation de simples réseaux de neurones tels qu'un perceptron afin d'identifier un caractère donné pourrait également servir de correcteur au KPPV pour les quelques erreurs remarquées.
