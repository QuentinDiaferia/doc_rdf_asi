\documentclass[a4paper,12pt]{article}
\usepackage[utf8]{inputenc}
\usepackage[francais]{babel}
\usepackage[margin=2cm]{geometry}
\usepackage[T1]{fontenc}
\usepackage{graphicx}
\usepackage{listing}
\usepackage{listings}
\usepackage{color}

\definecolor{mygreen}{rgb}{0,0.6,0}
\definecolor{mygray}{rgb}{0.5,0.5,0.5}
\definecolor{mymauve}{rgb}{0.58,0,0.82}

\lstset{ %
  backgroundcolor=\color{white},   % choose the background color; you must add \usepackage{color} or \usepackage{xcolor}
  basicstyle=\footnotesize,        % the size of the fonts that are used for the code
  breakatwhitespace=false,         % sets if automatic breaks should only happen at whitespace
  breaklines=true,                 % sets automatic line breaking
  captionpos=b,                    % sets the caption-position to bottom
  commentstyle=\color{mygreen},    % comment style
  deletekeywords={...},            % if you want to delete keywords from the given language
  escapeinside={\%*}{*)},          % if you want to add LaTeX within your code
  extendedchars=true,              % lets you use non-ASCII characters; for 8-bits encodings only, does not work with UTF-8
  frame=single,                    % adds a frame around the code
  keepspaces=true,                 % keeps spaces in text, useful for keeping indentation of code (possibly needs columns=flexible)
  keywordstyle=\color{blue},       % keyword style
  language=Matlab,                 % the language of the code
  morekeywords={*,...},            % if you want to add more keywords to the set
  numbers=left,                    % where to put the line-numbers; possible values are (none, left, right)
  numbersep=5pt,                   % how far the line-numbers are from the code
  numberstyle=\tiny\color{mygray}, % the style that is used for the line-numbers
  rulecolor=\color{black},         % if not set, the frame-color may be changed on line-breaks within not-black text (e.g. comments (green here))
  showspaces=false,                % show spaces everywhere adding particular underscores; it overrides 'showstringspaces'
  showstringspaces=false,          % underline spaces within strings only
  showtabs=false,                  % show tabs within strings adding particular underscores
  stepnumber=2,                    % the step between two line-numbers. If it's 1, each line will be numbered
  stringstyle=\color{mymauve},     % string literal style
  tabsize=2,                       % sets default tabsize to 2 spaces
  title=\lstname                   % show the filename of files included with \lstinputlisting; also try caption instead of title
}

\title{TP Document - Reconnaissance de formes}

\author{Quentin Diaferia ~~\\ Florian Lepetit}

\begin{document}

\maketitle

\begin{center}
%\includegraphics[width=0.7\textwidth]{images/logo.jpg}
\end{center}

\newpage

\tableofcontents

\newpage

\section{Introduction}

\paragraph{}
L'objectif de ce TP est de mettre en place un système de reconnaissance de formes capable de différencier les dix chiffres de 0 à 9 écrits de façon manuscrite. 

\paragraph{}
Pour cela, nous allons compléter un programme Matlab suivant deux étapes : apprentissage et prise de décision. Nous étudierons ensuite les performances de ce 
programme et en discuterons les améliorations possibles. 


\newpage

\section{Apprentissage}

\paragraph{}
Cette étape consisite à fournir au programme un jeu de caractères où chaque caractère est identifié : le programme sait s'il a affaire à un 0, un 1, etc. 
Ce jeu va lui permettre de reconnaître les caractéristiques de chaque caractère et donc de les reconnaître par la suite. La partie du programme extrayant 
les différents caractères de l'image est fournie par le sujet. 

\paragraph{}
Nous avons choisi de nous intéresser à la méthode du zoning utilisant les densités de pixels. Pour cela, nous avons développé la fonction extraitDensite 
ci-dessous.
\\
\begin{lstlisting}
function [ densites ] = extraitDensite(img, n, m)
	% On cree un vecteur qui contiendra les densites de chaque zone.
	densites = zeros(n*m, 1);
	% On cree une seconde image dont les dimensions sont corrigees afin de correspondre au nombre de zones demandees.
	[lignes1, colonnes1] = size(img);
	pasLignes = ceil(lignes1/n);
	pasColonnes = ceil(colonnes1/m);
	% On remplit cette image de blanc.
	imgCor = ones(pasLignes*n, pasColonnes*m);
	imgCor = 255*imgCor;
	% On copie les valeurs de l'image initiale dans la nouvelle.
	for i=1:lignes1
		for j=1:colonnes1
			imgCor(i,j) = img(i, j);
		end
	end
	% Initialisation des compteurs pour le calcul de la densite.
	i = 1;
	j = 1;
	iDen = 1;
	for i=1:pasLignes:pasLignes*n
		for j=1:pasColonnes:pasColonnes*m
			% On extrait la zone de l'image a traiter.
			img2 = subimage(imgCor, pasColonnes-1, pasLignes-1, j, i);
			% Calcul de la densite.
			densites(iDen) = sum(sum(img2)) / (pasColonnes*pasLignes*255);
			iDen = iDen+1;
		end
	end
end
\end{lstlisting}

\paragraph{}
Cette fonction retourne un vecteur de taille n * m contenant les densités de chacune des n * m zones de l'image. Pour les calculer, une deuxième image 
est créée dont les dimensions permettent une division en n * m zones. Pour cela, des pixels blancs sont ajoutés en bas et à droite de l'image. Cette pratique 
cause une légère perte de précision du calcul, mais le simplifie notablement.

\paragraph{}
On applique ensuite la fonction subimage pour découper la zone de l'image à traiter, et on calcule la densité de cette zone par une simple somme des pixels 
de la zone. Cette densité est ensuite normalisée par une division par la surface de la zone multipliée par 255.

\paragraph{}
Ensuite, il nous faut déterminer le profil type de chaque caractère. Nous avons désormais dix densités pour chacun des dix chiffres. Nous avons donc 
décidé de calculer, pour chaque chiffre, la moyenne des densités calculées précédemment. Il nous suffira donc par la suite de comparer une nouvelle densité 
avec ces moyennes pour déterminer la classe d'un caractère.
\\
\begin{lstlisting}
moyenne = zeros(10,25);
for i=1:10
    moyenne(i,:) = mean(densites( ((i-1)*20+1):20*i ,:));
end
\end{lstlisting}

\paragraph{}
Le code \verb?densites(((i-1)*20+1):20*i ,:)? permet de sélectionner les dix densités appartenant à la même classe dans le vecteur densites.
La variable moyenne est ensuite enregistrée dans le disque dur afin d'être réutilisée dans la partie suivante.








\newpage

\section{Prise de décision}


\newpage

\section{Performances}


\newpage

\section{Améliorations possibles}


\newpage

\section{Conclusion}

\paragraph{}
Malgré un taux de reconnaissance passable et des délais d'exécution corrects, la méthode du KPPV retenue peut se voir améliorer afin d'optimiser les performances du programme. Aussi, l'utilisation de simples réseaux de neurones tels qu'un perceptron afin d'identifier un caractère donné pourrait également servir de correcteur au KPPV pour les quelques erreurs remarquées.


\end{document}
